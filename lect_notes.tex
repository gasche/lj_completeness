% to compile this file, you need the 'mathpartir' package
% from Didier Rémy : http://gallium.inria.fr/~remy/latex/

\documentclass{article} 

\usepackage{times}
\usepackage{amsthm}
\usepackage{amsmath}
\usepackage{amssymb}
\usepackage{hyperref}
\usepackage{fullpage}
\usepackage{moreverb}
\usepackage{color}
\usepackage{ifthen}
\usepackage{supertabular}
\usepackage{mathpartir}

\begin{document}
\bibliographystyle{plain}
\newtheorem{thm}{Theorem}
\newtheorem{lemma}[thm]{Lemma}

\newtheorem{corollary}[thm]{Corollary}
\newtheorem{definition}[thm]{Definition}
\newtheorem{remark}[thm]{Remark}
\newtheorem{proposition}[thm]{Proposition}
\newtheorem{notn}[thm]{Notation}
\newtheorem{observation}[thm]{Observation}

\newcommand{\interp}[1]{[\negthinspace[#1]\negthinspace]}
\newcommand{\normto}[0]{\rightsquigarrow^{!}}
\newcommand{\case}[4]{\text{case}\ #1\ \text{of}\ #2\text{.}#3\text{,}#2\text{.}#4}
\newcommand{\join}[0]{\downarrow}

\title{Lecture Note: Proving Completeness of LJ}
\author{Harley D. Eades III \\
Computer Science\\
The University of Iowa}

\maketitle

\section{Introduction}
\label{sec:introduction}
In this short lecture note we show how to prove completeness for the
multi-succedent intuitionistic propositional logic LJ. The proof
method we use is due to Mints in his book ``A Short Introduction to
Intuitionistic Logic.'' The proof of completeness is quite remarkable.
All the definitions we will develop fit together beautifully.  The usual
statement of completeness is the following:
\begin{center}
  \begin{tabular}{lll}
    If a formula is valid then it is derivable in the logic. & (1)
  \end{tabular}
\end{center}
The model we use is the standard Kripke semantics for intuitionistic logic.
Details of this model can be found in Chap. 7 of Mints' book.  Now instead of proving
(1) directly we instead prove the contrapositive. That is 
\begin{center}
  \begin{tabular}{lll}
    \begin{tabular}{lll}
      For every underivable formula there exists a model such that the formula is falsified. & (2)
    \end{tabular}
  \end{tabular}
\end{center}
It turns out that we can show that every underivable formula is falsified in the canonical model.  So our
theorem statement becomes the following:
\begin{center}
  \begin{tabular}{lll}
    \begin{tabular}{lll}
      Every underivable formula is falsified in the canonical model. & (3)
    \end{tabular}
  \end{tabular}
\end{center}
Our first step to proving (3) is to define the logic then the canonical model.  Following this we will show
that all underivable sequents are falsified in the canonical model.  Then using this fact we will conclude completeness.
% section introduction (end)

\section{The Logic LJ}
\label{sec:the_logic_lj}
In this section we define the logic which we will prove complete.  The logic is a version of
Gentzen's multi-succedent intuitionistic propositional logic. The inference rules are defined
in Figure~\ref{fig:lj_ir}. This is the formulation due to 
Mints, however, we modified the axiom inference rule.  Our version results in shorter proof derivations. 
The rule Mints uses is the following:
\begin{center}
  \begin{math}
    $$\mprset{flushleft}
    \inferrule* [right=MintsAx] {
      \ 
    }{\phi \vdash \phi}
  \end{math}
\end{center}
In order to apply his rule one must use the weaking rules to remove any other hypotheses on the left and
any other conclusions on the right.  Thus, proof trees will be bigger.  It may be the case that Mints' 
rule is more amendable to proof search, but this does not concern us here.

\begin{figure}[h]
  \begin{center}
    \begin{math}
      \begin{array}{ccc}
        $$\mprset{flushleft}
        \inferrule* [right=Ax] {
          \ 
        }{\Gamma, \phi, \Gamma' \vdash \Delta,\phi,\Delta'}
        &
        $$\mprset{flushleft}
        \inferrule* [right=$\perp$] {
          \ 
        }{\Gamma, \perp \vdash \Delta,p}\\
        & \\
        $$\mprset{flushleft}
        \inferrule* [right=$\land_\text{r}$] {
          \Gamma \vdash \Delta,\phi
          \\
          \Gamma \vdash \Delta,\psi
        }{\Gamma \vdash \Delta,\phi \land \psi}
        & 
        $$\mprset{flushleft}
        \inferrule* [right=$\land_\text{l}$] {
          \phi,\psi,\Gamma \vdash \Delta
        }{\phi \land \psi,\Gamma \vdash \Delta}\\
        & \\
        $$\mprset{flushleft}
        \inferrule* [right=$\lor_\text{r}$] {
          \Gamma \vdash \Delta,\phi,\psi
        }{\Gamma \vdash \Delta,\phi \lor \psi}
        &
        $$\mprset{flushleft}
        \inferrule* [right=$\lor_\text{l}$] {
          \phi,\Gamma \vdash \Delta
          \\
          \psi,\Gamma \vdash \Delta
        }{\phi \lor \psi, \Gamma \vdash \Delta}\\
        & \\
        $$\mprset{flushleft}
        \inferrule* [right=$\to_\text{r}$] {
          \phi,\Gamma \vdash \psi
        }{\Gamma \vdash \Delta,\phi \to \psi}
        &
        $$\mprset{flushleft}
        \inferrule* [right=$\to_\text{l}$] {
          \phi \to \psi,\Gamma \vdash \Delta,\phi
          \\
          \psi,\Gamma \vdash \Delta
        }{\phi \to \psi,\Gamma \vdash \Delta}\\
        & \\        
        $$\mprset{flushleft}
        \inferrule* [right=$\text{Contr}_\text{r}$] {
          \Gamma \vdash \Delta,\phi,\phi
        }{\Gamma \vdash \Delta,\phi}
        &
        $$\mprset{flushleft}
        \inferrule* [right=$\text{Contr}_\text{l}$] {
          \phi,\phi,\Gamma \vdash \Delta
        }{\phi,\Gamma \vdash \Delta}        \\
        &\\
        $$\mprset{flushleft}
        \inferrule* [right=$\text{Weak}_\text{r}$] {
          \Gamma \vdash \Delta
        }{\Gamma \vdash \Delta,\phi}        
        &
        $$\mprset{flushleft}
        \inferrule* [right=$\text{Weak}_\text{l}$] {
          \Gamma \vdash \Delta
        }{\phi,\Gamma \vdash \Delta}
      \end{array}
    \end{math}
  \end{center}
  \caption{Gentzen's LJ}
  \label{fig:lj_ir}
\end{figure}

% section the_logic_lj (end)


\section{The Canonical Model}
\label{sec:the_canonical_model}
We are now ready to work towards the definition of the canonical
model. However, before we can define it we will need some mathematical
apparatus first. We call a sequent \textbf{underivable} if and only if
there does not exist a derivation in the logic starting with the
sequent. To define the canonical model we first must define when a
sequent is full, and when a sequent is saturated for invertible rules.
We denote the set of subformulas of a set of formulas $\Gamma$ as
$Sub(\Gamma)$, and we extend this to multiple sets of formulas by
$Sub(\Gamma,\Delta) = Sub(\Gamma) \cup Sub(\Delta)$.

\begin{definition}[Formula Full Sequent]
  \label{def:full_sequent_formula}  
  A sequent $\Gamma \vdash \Delta$ is \textbf{full for a formula} $\phi \in Sub(\Gamma,\Delta)$ if and only if 
  it is underivable and either $\phi \in \Gamma \cup \Delta$ or $\Gamma \vdash \Delta,\phi$ and $\phi,\Gamma \vdash \Delta$
  are derivable.
\end{definition}

\begin{definition}[Full Sequent]
  \label{def:full_sequent}
  A sequent $\Gamma \vdash \Delta$ is \textbf{full} if and only if it is formula full for all formulas in 
  $Sub(\Gamma,\Delta)$.
\end{definition}
Mints calls full sequents complete.  He also does not split the definition up as we do here.  Nonetheless he makes use
of the previous definition without defining it.  The key property of this entire proof is the notion of full sequents.
Because we will show that all the worlds of the canonical model are full.  Thus, to show that all underivable sequents
are falsified in the model then we better have a way to make any underivable sequent full.  This point is the guiding 
reason why we need all of this mathematical apparatus.  The next definition defines when a sequent is saturated for
invertible rules.  

\begin{definition}[Saturated for Invertible Rules]
  \label{def:sat_for_invert_rules}
  A sequent $\Gamma \vdash \Delta$ is \textbf{saturated for invertible rules} if it is
  underivable and the following conditions hold for any $\phi$ and $\psi$:
    \begin{itemize}
    \item[Case.] If $\phi \land \psi \in \Delta$ then $\phi \in \Delta$ or $\psi \in \Delta$.
    \item[Case.] If $\phi \land \psi \in \Gamma$ then $\phi \in \Gamma$ and $\psi \in \Gamma$.
    \item[Case.] If $\phi \lor \psi \in \Delta$ then $\phi \in \Delta$ and $\psi \in \Delta$.
    \item[Case.] If $\phi \lor \psi \in \Gamma$ then $\phi \in \Gamma$ or $\psi \in \Gamma$.      
    \item[Case.] If $\phi \to \psi \in \Gamma$ then $\phi \in \Delta$ or $\psi \in \Gamma$.      
    \end{itemize}    
\end{definition}
\noindent
Notice in the above definition that we do not have the following condition:
\begin{center}
  If $\phi \to \psi \in \Delta$ then $\phi \in \Gamma$ and $\psi \in \Delta$.      
\end{center}
This is because the rule for right implication is not an invertible rule.  

\noindent
The following result holds for full sequents.
\begin{lemma}[Saturation]
  \label{lemma:saturation}
  If $\Gamma \vdash \Delta$ is full, then it is saturated for invertible rules.
\end{lemma}
If one stops and ponders what the previous statement says it makes perfect sense.  A full
sequent is one in which every subformula of sequent is contained in either the left of the
sequent or the right depending on underivability. This definition lends itself to the 
fact that such a sequent is saturated for invertible rules.  The proof of the previous
lemma can be found in the book p. 58.

The next definition defines the completion sequences for an underivable sequent.  These 
sequences will be used to construct an equivalent full sequent of some underivable sequent.
Again, being able to do this is key for the proof of completeness.
\begin{definition}[The Completion Sequences]
  \label{def:completion_sequence}
  Suppose $\Gamma_0 \vdash \Delta_0$ is an underivable sequent, and that
  $\phi_0, \ldots, \phi_n$ is a fixed sequence of all the formulas in $Sub(\Gamma_0,\Delta_0)$.  
  Then we define the \textbf{completion} sequences for $\Gamma_0 \vdash \Delta_0$ by constructing the sequences
  $\Gamma_0' \subseteq \cdots \subseteq \Gamma_{n+1}'$  and $\Delta_0' \subseteq \cdots \subseteq \Delta_{n+1}'$
  of finite sets of formulas such that $\Gamma_i' \vdash \Delta_i'$ is underivable and full for 
  formulas $\phi_j$ for all $j < i$.  That is either $\phi_j \in \Gamma_i' \cup \Delta_i'$ or both
  $\Gamma_i' \vdash \Delta_i',\phi_j$ and $\phi_j,\Gamma_i' \vdash \Delta_i'$ are derivable.

  \ \\
  The construction is defined by mutual recursion as follows:
  \begin{center}
    \begin{math}
      \begin{array}{llll}
        \Gamma'_0         & := & \Gamma_0 \\
        \Gamma'_{i+1}     & := & \phi_i,\Gamma'_i & \text{ if } \phi_i, \Gamma'_i \vdash \Delta'_i 
                                                    \text{ is underivable }\\
        \Gamma'_{i+1}     & := & \Gamma'_i & \text{ otherwise }
      \end{array}
    \end{math}
  \end{center}
  \noindent
  and \ \\
  \begin{center}
    \begin{math}
      \begin{array}{llll}
        \Delta'_0         & := & \Delta_0 \\
        \Delta'_{i+1}     & := & \phi_i,\Delta'_i & \text{ if } \Gamma'_{i+1} \vdash \Delta'_i,\phi_i
                                                    \text{ is underivable }\\
        \Delta'_{i+1}     & := & \Delta'_i & \text{ otherwise. }
      \end{array}
    \end{math}
  \end{center}
\end{definition}
\noindent
Naturally, we show that each sequent built from the completion sequences are full.  The proof of this
lemma is non-trivial.  It is easy, but non-trivial.  Mints however does not give a proof of this, but
rather assumes it.  In fact Mints rolls the previous definition and the next two lemmas into
one lemma. We believe these are in error and aids confusion.

\begin{lemma}
  \label{lemma:completion_aux}
  If $\Gamma_0 \subseteq \cdots \subseteq \Gamma_n$ and $\Delta_0 \subseteq \cdots \subseteq \Delta_n$ are
  completion sequences with respect to the sequence of subformulas $\phi_0, \ldots , \phi_{n-1}$ then 
  $\Gamma_i \vdash \Delta_i$ is full with respect to the formulas $\phi_j$ for all $j < i$.
\end{lemma}
\begin{proof}
  This is a proof by induction on $i$. 
  \begin{itemize}
  \item[Base Case.] Then we must show that $\Gamma_0 \vdash \Delta_0$ is full with respect to
    the formulas $\phi_j$ for all $j < 0$.  This is trivially the case.

  \item[Step Case.] We must show that $\Gamma_{i+1} \vdash \Delta_{i+1}$ is full with respect to the formulas
    $\phi_j$ for all $j < i+1$.  Based on the definition of completion sequences we know one of two things
    about $\Gamma$.
    \begin{itemize}
    \item[Case.] Suppose $\Gamma_{i+1} = \Gamma_i,\phi_i$.  Then it must be the case that 
      $\Gamma_{i+1} \vdash \Delta_{i}$ is underivable.  Now we case split on $\Delta$.
      \begin{itemize}
      \item[Case.] Suppose $\Delta_{i+1} = \Delta_i,\phi_i$.  Then it must be the case that 
        $\Gamma_{i+1} \vdash \Delta_i,\phi_i$ is underivable.  However, this is not the case.
        Thus we have arrived at a contradiction.

      \item[Case.] Suppose $\Delta_{i+1} = \Delta_{i}$.  Fullness of $\Gamma_{i+1} \vdash \Delta_{i+1}$
        follows from the fact that we know by the IH that $\Gamma_i \vdash \Delta_i$ is full with respect
        to formulas $\phi_j$ for all $j < i$, $\phi_i \in \Gamma_i \cup \Delta_i$, and $\Gamma_{i+1} \vdash \Delta_i$
        is underivable.
      \end{itemize}

    \item[Case.] Suppose $\Gamma_{i+1} = \Gamma_i$.  We case split on the form of $\Delta$.
      \begin{itemize}
      \item[Case.] Suppose $\Delta_{i+1} = \Delta_i,\phi_i$.  Then it must be the case that 
        $\Gamma_{i+1} \vdash \Delta_{i+1}$ is underivable.  In addition we know that 
        $\phi_i \in \Gamma_{i+1} \cup \Delta_{i+1}$.  By the IH we know that $\Gamma_i \vdash \Delta_i$ is
        full with respect to the formulas $\phi_j$ for all $j < i$.  Thus, by definition we know
        $\Gamma_{i+1} \vdash \Delta_{i+1}$ is also full.
        
      \item[Case.] Suppose $\Delta_{i+1} = \Delta_{i}$.  Fullness of $\Gamma_{i+1} \vdash \Delta_{i+1}$
        follows from the fact that we know $\Gamma_i,\phi_i \vdash \Delta_i$ and
        $\Gamma_i \vdash \Delta_i,\phi_i$ are derivable.  
      \end{itemize}

    \end{itemize}

  \end{itemize}

\end{proof}
\noindent
Finally, the main completion lemma.
\begin{lemma}[Completion]
  \label{lemma:completion}
  Any underivable sequent $\Gamma_0 \vdash \Delta_0$ may be extended to a full sequent
  $\Gamma \vdash \Delta$ for some $\Gamma$ and $\Delta$ consisting of subformulas of 
  $\Gamma_0$ and $\Delta_0$.
\end{lemma}
\begin{proof}
  Using Definition~\ref{def:completion_sequence} and Lemma~\ref{lemma:completion_aux} we 
  can construct the completion sequence $\Gamma_0 \subseteq \cdots \subseteq \Gamma_n$ and
  $\Delta_0 \subseteq \cdots \subseteq \Delta_n$ with respect to some sequence of
  the formulas in $Sub(\Gamma_0,\Delta_0)$, $\phi_0,\ldots ,\phi_{n-1}$, such that 
  $\Gamma_i \vdash \Delta_i$ is formula full with respect to the formulas $\phi_j$ for all
  $j < i$.  Take $\Gamma = \Gamma_n$ and $\Delta = \Delta_n$.  
  Clearly, $\Gamma \vdash \Delta$ is full.  
\end{proof}

Our proof of the completion lemma is a bit different from Mints.  He defines $\Gamma = \cup \Gamma_i$ and
$\Delta = \cup\Delta_i$.  However, each $\Gamma_i$ and $\Delta_i$ are finite sets of formulas, and 
each $\Gamma_i \subseteq \Gamma_{i+1}$ and $\Delta_i \subseteq \Delta_{i+1}$.  Thus, $\Gamma_n$ and
$\Delta_n$ will contain all of the previous elements.  Thus, the union is not needed. Now Mints' definition
would be correct if we were using possibly infinite sets, but this is not the case.
We now arrive at the definition of the canonical Kripke model. This
will be the counter model we will use to show completeness.

\begin{definition}[Canonical Model]
  \label{def:canon_model}
  The \textbf{canonical Kripke model} $K$ is a tuple $\langle W, R_\subseteq, V_{\in} \rangle$
  such that 
  \begin{center}
    \begin{itemize}
    \item $W$ is the set of all full sequents,
    \item $R_\subset(\Gamma \vdash \Delta,\Gamma' \vdash \Delta') = \Gamma \subseteq \Gamma'$, and
    \item $V_\in(p, \Gamma \vdash \Delta) = p \in \Gamma$.
    \end{itemize}
  \end{center}
\end{definition}

Now that we defined the canonical model our goal is to show that every underivable sequent is falsified
in the canonical model. We first define what we mean by falsified in a model.
\begin{definition}
  \label{def:Falsification}
  A sequent $\Gamma \vdash \Delta$ is \textbf{falsified} in a world $w$ of a Kripke model if
  $V(\land \Gamma,w) = 1$ and $V(\lor \Delta, w) = 0$.  This implies that $V(\Gamma \vdash \Delta, w) = 0$.
\end{definition}

Above we saw when a sequent is saturated for invertible rules.  Next we define when a set of sequents 
is saturated for non-invertible rules. Following this definition is the definition of when a set is
saturated.
\begin{definition}[Saturated for Non-Invertible Rules]
  \label{def:sat_non-invert_rules}
  A set of sequents $M$ is \textbf{saturated for non-invertible rules} if the
  following condition is satisfied for every $\Gamma \vdash \Delta$
  in $M$:
  \begin{center}
    if $\phi \to \psi \in \Delta$, then there is a sequent $\Gamma' \vdash \Delta' \in M$ such that
    $\phi,\Gamma \subseteq \Gamma'$ and $\psi \in \Delta'$.
  \end{center}
\end{definition}

\begin{definition}[Saturated Set]
  \label{def:sat_set}
  A set of sequents $M$ is \textbf{saturated} if every sequent in $M$ is saturated for invertible rules
  and $M$ is saturated for non-invertible rules.
\end{definition}

\noindent
It turns out that the canonical model is saturated.  

\begin{lemma}[Canonical Model is Saturated]
  \label{lemma:canonical_model_is_saturated}
  The set $W$ of the canonical model is saturated.
\end{lemma}
\begin{proof}
  By definition of $W$ every sequent in $W$ is full, hence by Lemma~\ref{lemma:saturation} they are 
  saturated for invertible rules.  It suffices to show that $W$ is saturated for non-invertible rules.
  If $\Gamma \vdash \phi \to \psi,\Delta \in W$ then it must be the case that $\phi,\Gamma \vdash \psi,\Delta$
  is underivable.  So by Lemma~\ref{lemma:completion} we may construct the full sequent 
  $\Gamma' \vdash \Delta' \in W$  for some $\Gamma'$ and $\Delta'$, such that $\phi,\Gamma \subseteq \Gamma'$ and 
  $\psi \in \Delta'$.
\end{proof}

\noindent
Finally we show our main theorem.  
\begin{thm}[Falsification in $K$]
  \label{thm:falsification_in_k}
  Let $K = \langle W, R_\subseteq, V_\in \rangle$ be the canonical model. Then for $w \equiv \Gamma \vdash \Delta \in W$:
  \begin{center}
    \begin{itemize}
    \item[a.] If $\phi \in \Gamma$ then $V_\in(\phi, w) = 1$, and
    \item[b.] if $\phi \in \Delta$ then $V_\in(\phi,w) = 0$.
    \item[c.] This implies that $V_\in(\Gamma \vdash \Delta, w) = 0$, that is, $w$ is 
      falsified in $K$.
    \end{itemize}
  \end{center}  
\end{thm}
\begin{proof}
  The conditions a and b imply the condition c. Thus it suffices to show conditions 
  a and b. They can be proven by simultaneous induction on $\phi$. 
  \begin{itemize}
  \item[Case.] Suppose $\phi \equiv p$.  
    \begin{itemize}
    \item[a.] If $\phi \in \Gamma$ then by the definition of $K$ we have 
      $V_\in(\phi,w) = 1$.
    \item[b.] If $\phi \in \Delta$ then it must be the case
    that $\phi \not\in \Gamma$ or $\Gamma \vdash \Delta$ would not be full (because it
    is derivable), but all worlds of $K$ are full.  Therefore, it must be the case
    that $V_\in(\phi,w) = 0$.
    \end{itemize}

    
  \item[Case.] Suppose $\phi \equiv \phi_1 \land \phi_2$.
    \begin{itemize}
    \item[a.] Suppose $\phi \in \Gamma$.  Then we know $\phi_1 \in \Gamma$
      and $\phi_2 \in \Gamma$, because $\Gamma \vdash \Delta$ is full, hence,
      saturated for invertible rules. Thus, $V_\in(\phi_1,w) = 1$ and $V_\in(\phi_2,w) = 1$
      by the IH.
      
    \item[b.] Suppose $\phi \in \Delta$. Then it must be the case that 
      either $\phi_1 \in \Delta$ or $\phi_2 \in \Delta$ because $\Gamma \vdash \Delta$
      is saturated for invertible rules.  Consider the former.  Then by the IH 
      $V_\in(\phi_1,w) = 0$.  This implies that $V_\in(\phi,w) = 0$.  The case 
      where $\phi_2 \in \Delta$ is similar.
    \end{itemize}

  \item[Case.] Suppose $\phi \equiv \phi_1 \lor \phi_2$.  Similar to the case for conjunction.
    
  \item[Case.] Suppose $\phi \equiv \phi_1 \to \phi_2$.
    \begin{itemize}
    \item[a.] If $\phi \in \Gamma$ then for every $w' = \Gamma' \vdash \Delta'$ such that $R_W(w,w')$ ($\Gamma \subseteq \Gamma'$)
      we have $\phi \in \Gamma'$.  By saturation for invertible rules we have $\phi_1 \in \Delta'$ or $\phi_2 \in \Gamma'$.  
      Thus by the IH we have $V_\in(\phi_1,w') = 0$ or $V_\in(\phi_2,w') = 1$.  This implies that $V_\in(\phi,w) = 1$.

    \item[b.] If $\phi \in \Delta$ then by Lemma~\ref{lemma:canonical_model_is_saturated} $W$ is saturated for non-invertible
      rules.  Hence there exists a $w' = \Gamma' \vdash \Delta'$ such that $R_W(w,w')$, $\phi_1 \in \Gamma'$ and $\phi_2 \in \Delta'$.
      By the IH $V_\in(\phi_1,w') = 1$ and $V_\in(\phi_2,w') = 0$.  This implies that $V_\in(\phi,w) = 0$.
    \end{itemize}    
  \end{itemize}
\end{proof}

\noindent
This is all that is needed to prove completeness of LJ.  We prove this in the next section.
% section the_canonical_model (end)

\section{Completeness}
\label{sec:completeness}
Completeness follows easily from our previous result.  
\begin{thm}[Completeness]
  \label{thm:completeness}
  Each sequent underivable in LJ is falsified in the canonical model $K$.  Hence
  every valid sequent is derivable in LJ.
\end{thm}
\begin{proof}
  Suppose that $\Gamma \vdash \Delta$ is an underivable sequent in LJ, and $W$ is the saturated set of all full sequents.
  That is $W$ is the set of worlds of $K$. Then by completion (Lemma~\ref{lemma:completion}),
  there exists a $\Gamma'$ and $\Delta'$ such that $\Gamma' \vdash \Delta'$ is the completed version of $\Gamma \vdash \Delta$. 
  Sense $W$ is saturated we know $\Gamma' \vdash \Delta' \in W$, and we may apply Theorem~\ref{thm:falsification_in_k} to obtain
  that $V_\in(\Gamma' \vdash \Delta', \Gamma' \vdash \Delta') = 0$.  Hence, 
  $V_\in(\Gamma \vdash \Delta, \Gamma' \vdash \Delta') = 0$, by the definition of validity of sequents.
\end{proof}
Mints shows soundness of our logic in Section 8.1 on page 57.  We can use it along with completeness to show admissibility of
cut.  
\begin{corollary}[Admissibility of Cut]
  \label{corollary:admissibility_of_cut}
  If $\Gamma \vdash \Delta, \phi$ and $\phi, \Gamma \vdash \Delta$ are derivable, then $\Gamma \vdash \Delta$ is derivable.
\end{corollary}
\begin{proof}
  Suppose $\Gamma \vdash \Delta, \phi$ and $\phi, \Gamma \vdash \Delta$ are derivable for some $\Gamma$,$\Delta$, and $\phi$.
  Clearly, $\Gamma \vdash \phi$ is derivable by a simple application of the cut rule.  Thus, by soundness + cut we know 
  $\Gamma \vdash \phi$ is valid.  Now by completeness (without cut) $\Gamma \vdash \Delta$ is derivable. 
\end{proof}
% section completeness (end)
\end{document}