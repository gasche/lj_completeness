\documentclass{article} 

\usepackage{times}
\usepackage{amsthm}
\usepackage{amsmath}
\usepackage{amssymb}
\usepackage{hyperref}
\usepackage{fullpage}
\usepackage{moreverb}
\usepackage{color}
\usepackage{ifthen}
\usepackage{supertabular}

\begin{document}
\bibliographystyle{plain}
\newtheorem{thm}{Theorem}
\newtheorem{lemma}[thm]{Lemma}

\newtheorem{corollary}[thm]{Corollary}
\newtheorem{definition}[thm]{Definition}
\newtheorem{remark}[thm]{Remark}
\newtheorem{proposition}[thm]{Proposition}
\newtheorem{notn}[thm]{Notation}
\newtheorem{observation}[thm]{Observation}

\newcommand{\interp}[1]{[\negthinspace[#1]\negthinspace]}
\newcommand{\normto}[0]{\rightsquigarrow^{!}}
\newcommand{\case}[4]{\text{case}\ #1\ \text{of}\ #2\text{.}#3\text{,}#2\text{.}#4}
\newcommand{\join}[0]{\downarrow}

\title{A Completeness Proof for LJ}
\author{Harley D. Eades III \\
Computer Science\\
The University of Iowa}

\maketitle

\section{Introduction}
\label{sec:introduction}
In this short lecture note we summarize how to show completeness for
the multi-succedent intuitionistic propositional logic.  The proof
method we use is due to Mints.  All the details can be found in his
book ``A Short Introduction to Intuitionistic Logic.'' 
% section introduction (end)

\section{The Canonical Model}
\label{sec:the_canonical_model}
In this section we define the canonical model and give some properties
of this model.  We call a sequent \textbf{underivable} if and only if
there does not exist a derivation in the logic starting with the
sequent.  

To define the canonical model we first must define when a sequent is
full, and when a sequent is saturated for invertible rules.  We
denote the set of subformulas of a set of formulas $\Gamma$ as
$Sub(\Gamma)$.

\begin{definition}[Formula Full Sequent]
  \label{def:full_sequent_formula}  
  A sequent $\Gamma \vdash \Delta$ is \textbf{full for a formula} $\phi \in Sub(\Gamma,\Delta)$ if and only if 
  it is underivable and either $\phi \in \Gamma \cup \Delta$ or $\Gamma \vdash \Delta,\phi$ and $\phi,\Gamma \vdash \Delta$
  are derivable.
\end{definition}

\begin{definition}[Full Sequent]
  \label{def:full_sequent}
  A sequent $\Gamma \vdash \Delta$ is \textbf{full} if and only if it is formula full for all formulas in 
  $Sub(\Gamma,\Delta)$.
\end{definition}

\begin{definition}[Saturated for Invertible Rules]
  \label{def:sat_for_invert_rules}
  A sequent $\Gamma \vdash \Delta$ is \textbf{saturated for invertible rules} if it is
  underivable and the following conditions hold for any $\phi$ and $\psi$:
    \begin{itemize}
    \item[Case.] If $\phi \land \psi \in \Delta$ then $\phi \in \Delta$ or $\psi \in \Delta$.
    \item[Case.] If $\phi \land \psi \in \Gamma$ then $\phi \in \Gamma$ and $\psi \in \Gamma$.
    \item[Case.] If $\phi \lor \psi \in \Delta$ then $\phi \in \Delta$ and $\psi \in \Delta$.
    \item[Case.] If $\phi \lor \psi \in \Gamma$ then $\phi \in \Gamma$ or $\psi \in \Gamma$.      
    \item[Case.] If $\phi \to \psi \in \Gamma$ then $\phi \in \Delta$ or $\psi \in \Gamma$.      
    \end{itemize}    
\end{definition}
\noindent
Notice in the above definition that we do not have the following condition:
\begin{center}
  If $\phi \to \psi \in \Delta$ then $\phi \in \Gamma$ and $\psi \in \Delta$.      
\end{center}
This is because the rule for right implication is not an invertible rule.  

\noindent
The following result holds for full sequents.
\begin{lemma}[Saturation]
  \label{lemma:saturation}
  If $\Gamma \vdash \Delta$ is full, then it is saturated for invertible rules.
\end{lemma}

It turns out that if a sequent is underivable then it may be extended into a new sequent
which remains underivable, but is full.  This is known as completion. 

\begin{definition}[The Completion Sequences]
  \label{def:completion_sequence}
  Suppose $\Gamma_0 \vdash \Delta_0$ is an underivable sequent, and that
  $\phi_0, \ldots, \phi_n$ is a fixed sequence of all the formulas in $Sub(\Gamma_0,\Delta_0)$.  
  Then we define the \textbf{completion} sequences for $\Gamma_0 \vdash \Delta_0$ by constructing the sequences
  $\Gamma_0' \subseteq \cdots \subseteq \Gamma_{n+1}'$  and $\Delta_0' \subseteq \cdots \subseteq \Delta_{n+1}'$
  of finite sets of formulas such that $\Gamma_i' \vdash \Delta_i'$ is underivable and full for 
  formulas $\phi_j$ for all $j < i$.  That is either $\phi_j \in \Gamma_i' \cup \Delta_i'$ or both
  $\Gamma_i' \vdash \Delta_i',\phi_j$ and $\phi_j,\Gamma_i' \vdash \Delta_i'$ are derivable.

  \ \\
  The construction is defined by mutual recursion as follows:
  \begin{center}
    \begin{math}
      \begin{array}{llll}
        \Gamma'_0         & := & \Gamma_0 \\
        \Gamma'_{i+1}     & := & \phi_i,\Gamma'_i & \text{ if } \phi_i, \Gamma'_i \vdash \Delta'_i 
                                                    \text{ is underivable }\\
        \Gamma'_{i+1}     & := & \Gamma'_i & \text{ otherwise }
      \end{array}
    \end{math}
  \end{center}
  \noindent
  and \ \\
  \begin{center}
    \begin{math}
      \begin{array}{llll}
        \Delta'_0         & := & \Delta_0 \\
        \Delta'_{i+1}     & := & \phi_i,\Delta'_i & \text{ if } \Gamma'_{i+1} \vdash \Delta'_i,\phi_i
                                                    \text{ is underivable }\\
        \Delta'_{i+1}     & := & \Delta'_i & \text{ otherwise. }
      \end{array}
    \end{math}
  \end{center}
\end{definition}

\begin{lemma}
  \label{lemma:completion_aux}
  If $\Gamma_0 \subseteq \cdots \subseteq \Gamma_n$ and $\Delta_0 \subseteq \cdots \subseteq \Delta_n$ are
  completion sequences with respect to the sequence of subformulas $\phi_0, \ldots , \phi_{n-1}$ then 
  $\Gamma_i \vdash \Delta_i$ is full with respect to the formulas $\phi_j$ for all $j < i$.
\end{lemma}
\begin{proof}
  This is a proof by induction on $i$. 
  \begin{itemize}
  \item[Base Case.] Then we must show that $\Gamma_0 \vdash \Delta_0$ is full with respect to
    the formulas $\phi_j$ for all $j < 0$.  This is trivially the case.

  \item[Step Case.] We must show that $\Gamma_{i+1} \Delta_{i+1}$ is full with respect to the formulas
    $\phi_j$ for all $j < i+1$.  Based on the definition of completion sequences we know one of two things
    about $\Gamma$.
    \begin{itemize}
    \item[Case.] Suppose $\Gamma_{i+1} = \Gamma_i,\phi_i$.  Then it must be the case that 
      $\Gamma_{i+1} \vdash \Delta_{i}$ is underivable.  Now we case split on $\Delta$.
      \begin{itemize}
      \item[Case.] Suppose $\Delta_{i+1} = \Delta_i,\phi_i$.  Then it must be the case that 
        $\Gamma_{i+1} \vdash \Delta_i,\phi_i$ is underivable.  However, this is not the case.
        Thus we have arrived at a contradiction.

      \item[Case.] Suppose $\Delta_{i+1} = \Delta_{i}$.  Fullness of $\Gamma_{i+1} \vdash \Delta_{i+1}$
        follows from the fact that we know by the IH that $\Gamma_i \vdash \Delta_i$ is full with respect
        to formulas $\phi_j$ for all $j < i$, $\phi_i \in \Gamma_i \cup \Delta_i$, and $\Gamma_{i+1} \vdash \Delta_i$
        is underivable.
      \end{itemize}

    \item[Case.] Suppose $\Gamma_{i+1} = \Gamma_i$.  We case split on the form of $\Delta$.
      \begin{itemize}
      \item[Case.] Suppose $\Delta_{i+1} = \Delta_i,\phi_i$.  Then it must be the case that 
        $\Gamma_{i+1} \vdash \Delta_{i+1}$ is underivable.  In addition we know that 
        $\phi_i \in \Gamma_{i+1} \cup \Delta_{i+1}$.  By the IH we know that $\Gamma_i \vdash \Delta_i$ is
        full with respect to the formulas $\phi_j$ for all $j < i$.  Thus, by definition we know
        $\Gamma_{i+1} \vdash \Delta_{i+1}$ is also full.
        
      \item[Case.] Suppose $\Delta_{i+1} = \Delta_{i}$.  Fullness of $\Gamma_{i+1} \vdash \Delta_{i+1}$
        follows from the fact that we know $\Gamma_i,\phi_i \vdash \Delta_i$ and
        $\Gamma_i \vdash \Delta_i,\phi_i$ are derivable.  
      \end{itemize}

    \end{itemize}

  \end{itemize}

\end{proof}

\begin{lemma}[Completion]
  \label{lemma:completion}
  Any underivable sequent $\Gamma_0 \vdash \Delta_0$ may be extended to a full sequent
  $\Gamma \vdash \Delta$ for some $\Gamma$ and $\Delta$ consisting of subformulas of 
  $\Gamma_0$ and $\Delta_0$.
\end{lemma}
\begin{proof}
  Using Definition~\ref{def:completion_sequence} and Lemma~\ref{lemma:completion_aux} we 
  can construct the completion sequence $\Gamma_0 \subseteq \cdots \subseteq \Gamma_n$ and
  $\Delta_0 \subseteq \cdots \subseteq \Delta_n$ with respect to some sequence of
  the formulas in $Sub(\Gamma_0,\Delta_0)$, $\phi_0,\ldots ,\phi_{n-1}$, such that 
  $\Gamma_i \vdash \Delta_i$ is formula full with respect to the formulas $\phi_j$ for all
  $j < i$.  Take $\Gamma = \Gamma_n$ and $\Delta = \Delta_n$.  
  Clearly, $\Gamma \vdash \Delta$ is full.  
\end{proof}

We now arrive at the definition of the canonical Kripke model. This
will be the counter model we will use to show completeness.

\begin{definition}[Canonical Model]
  \label{def:canon_model}
  The \textbf{canonical Kripke model} $K$ is a tuple $\langle W, R_\subseteq, V_{\in} \rangle$
  such that 
  \begin{center}
    \begin{itemize}
    \item $W$ is the set of all full sequents,
    \item $R_\subset(\Gamma \vdash \Delta,\Gamma' \vdash \Delta') = \Gamma \subseteq \Gamma'$, and
    \item $V_\in(p, \Gamma \vdash \Delta) = p \in \Gamma$.
    \end{itemize}
  \end{center}
\end{definition}

We now show that the canonical model falsifies every underivable sequent. 
\begin{definition}
  \label{def:Falsification}
  A sequent $\Gamma \vdash \Delta$ is \textbf{falsified} in a world $w$ of a Kripke model if
  $V(\land \Gamma,w) = 1$ and $V(\lor \Delta, w) = 1$.  This implies that $V(\Gamma \vdash \Delta, w) = 0$.
\end{definition}

Above we saw when a sequent is saturated for invertible rules.  Next we define when a set of sequents 
is saturated for non-invertible rules. Following this definition is the definition of when a set is
saturated.
\begin{definition}[Saturated for Non-Invertible Rules]
  \label{def:sat_non-invert_rules}
  A set of sequents $M$ is \textbf{saturated for non-invertible rules} if the
  following condition is satisfied for every $\Gamma \vdash \Delta$
  in $M$:
  \begin{center}
    if $\phi \to \psi \in \Delta$, then there is a sequent $\Gamma' \vdash \Delta' \in M$ such that
    $\phi,\Gamma \subseteq \Gamma'$ and $\psi \in \Delta$.
  \end{center}
\end{definition}

\begin{definition}[Saturated Set]
  \label{def:sat_set}
  A set of sequents $M$ is \textbf{saturated} if every sequent in $M$ is saturated for invertible rules
  and $M$ is saturated for non-invertible rules.
\end{definition}

The following result is important for completeness.

\begin{lemma}[Canonical Model is Saturated]
  \label{lemma:canonical_model_is_saturated}
  The set $W$ of the canonical model is saturated.
\end{lemma}
\begin{proof}
  By definition of $W$ every sequent in $W$ is full, hence by Lemma~\ref{lemma:saturation} they are 
  saturated for invertible rules.  It suffices to show that $W$ is saturated for non-invertible rules.
  If $\Gamma \vdash \phi \to \psi,\Delta \in W$ then it must be the case that $\phi,\Gamma \vdash \psi,\Delta$
  is underivable.  So by Lemma~\ref{lemma:completion} we may construct the full sequent 
  $\Gamma' \vdash \Delta' \in W$  for some $\Gamma'$ and $\Delta'$, such that $\phi,\Gamma \subseteq \Gamma'$ and 
  $\psi \in \Delta'$.
\end{proof}

\begin{thm}[Falsification in $K$]
  \label{thm:falsification_in_k}
  Let $K = \langle W, R_\subseteq, V_\in \rangle$ be the canonical model. Then for $w \equiv \Gamma \vdash \Delta \in W$:
  \begin{center}
    \begin{itemize}
    \item[] If $\phi \in \Gamma$ then $V_\in(\phi, w) = 1$, and
    \item[] if $\phi \in \Delta$ then $V_\in(\phi,w) = 0$.
    \end{itemize}
  \end{center}
  This implies that $V_\in(\Gamma \vdash \Delta, w) = 0$, that is, $w$ is falsified in $K$.
\end{thm}
\begin{proof}
  TODO.
\end{proof}

\noindent
This is all that is needed to prove completeness of LJ.  We prove this in the next section.
% section the_canonical_model (end)

\section{Completeness}
\label{sec:completeness}

\begin{thm}[Completeness]
  \label{thm:completeness}
  Each sequent underivable in LJ is falsified in the canonical model $K$.  Hence
  every valid sequent is derivable in LJ.
\end{thm}
\begin{proof}
  Suppose that $\Gamma \vdash \Delta$ is an underivable sequent in LJ, and $W$ is the saturated set of all full sequents.
  That is $W$ is the set of worlds of $K$. Then by completion (Lemma~\ref{lemma:completion}),
  there exists a $\Gamma'$ and $\Delta'$ such that $\Gamma' \vdash \Delta'$ is the completed version of $\Gamma \vdash \Delta$. 
  Sense $W$ is saturated we know $\Gamma' \vdash \Delta' \in W$, and we may apply Theorem~\ref{thm:falsification_in_k} to obtain
  that $V_\in(\Gamma' \vdash \Delta', \Gamma' \vdash \Delta') = 0$.  Hence, by monotonicity 
  $V_\in(\Gamma \vdash \Delta, \Gamma' \vdash \Delta') = 0$.
\end{proof}

\begin{corollary}[Admissibility of Cut]
  \label{corollary:admissibility_of_cut}
  If $\Gamma \vdash \Delta, \phi$ and $\phi, \Gamma \vdash \Delta$ are derivable, then $\Gamma \vdash \Delta$ is derivable.
\end{corollary}
\begin{proof}
  Suppose $\Gamma \vdash \Delta, \phi$ and $\phi, \Gamma \vdash \Delta$ are derivable for some $\Gamma$,$\Delta$, and $\phi$.
  Clearly, $\Gamma \vdash \phi$ is derivable by a simple application of the cut rule.  Thus, by soundness + cut we know 
  $\Gamma \vdash \phi$ is valid.  Now by completeness (without cut) $\Gamma \vdash \Delta$ is derivable. 
\end{proof}
% section completeness (end)

\end{document}